% IACR Transactions CLASS DOCUMENTATION -- version 0.24 (26 August 2016)
% Written by Gaetan Leurent gaetan.leurent@inria.fr (2016)
%
% To the extent possible under law, the author(s) have dedicated all
% copyright and related and neighboring rights to this software to the
% public domain worldwide. This software is distributed without any
% warranty.
%
% You should have received a copy of the CC0 Public Domain Dedication
% along with this software. If not, see
% <http://creativecommons.org/publicdomain/zero/1.0/>.


% Personal Note: Replace submission for preprint
\documentclass[preprint]{iacrtrans}
\usepackage[utf8]{inputenc}
\usepackage[flushleft]{threeparttable}
\usepackage[justification=centering]{caption}
\usepackage{tikz}
\usepackage{varwidth}
\usepackage[toc,page]{appendix}
\usepackage{float}
%\usepackage{algorithm2e}

\usepackage[backend=biber, ]{biblatex}
\addbibresource{library.bib}
\addbibresource{local.bib}

\usepackage{algorithm}
%\usepackage{algpseudocode}
\usepackage{algorithmic}
\usetikzlibrary{shapes.geometric,
                arrows,
                fit,         % For easily fitting nodes inside another one
                decorations.pathreplacing,
                calligraphy,
                positioning, % For easy node-relative placements
                calc
               }

\PassOptionsToPackage{hyphens}{url}
\usepackage{hyperref}

\author{}
\institute{}
%\institute{HasLAB, Departamento de Informática, Universidade do Minho \email{henriquejosefaria@icloud.com, jmvalenca@di.uminho.pt}}
\title[\texttt{High-Assurance Post-Quantum Cryptography and Distributed Secure Computation}]
{High-Assurance Post-Quantum Cryptography and Distributed Secure Computation \\ (work plan)}
\author{Henrique José Faria}

% subtitulo - an hybrid of PICNIC and SPHINCS+


% Extra tikz definitions

\tikzstyle{arrow} = [thick,->,>=stealth]
\tikzstyle{node} = [circle, draw, minimum size=4pt, inner sep=0pt, outer sep=0pt]
\tikzstyle{triangle} = [draw,regular polygon, regular polygon sides=3,minimum height=3cm]
\tikzstyle{block} = [rectangle, rounded corners, minimum width=3cm, minimum height=1cm,text centered, draw=black]

\newcommand*{\rom}[1]{\MakeUppercase{\romannumeral#1}}

\begin{document}

\maketitle

% use optional argument because the \LaTeX command breaks the PDF keywords
\keywords[High-Assurance Cryptographic Software, Program Verification, Post-Quantum Cryptography, Secure Distributed Computation]{High-Assurance Cryptographic Software \and Program Verification \and Post-Quantum Cryptography \and Secure Distributed Computation.}

\begin{abstract}
  Computer Aided Cryptography (CAC) aims to develop tools for the
  analysis and implementation of cryptographic protocols, including
  specification and implementation languages, compilers and formal
  verification tools. The formal verification dimension opens the way
  for High-Assurance Cryptographic Software. The objective of this
  work is to progress beyond the state of the art at both the
  foundational and applied research levels by developing new
  extensions to the Jasmin tool-chain and demonstrating its
  applicability to concrete examples that can have real-world impact.

  A recent SoK paper \cite{SoK} published at IEEE S\&P 2021
  summarises challenges in this area; two of these challenges serve as
  our main motivation: 1) removing scalability bottlenecks in the
  analysis of distributed cryptographic protocols and 2) enabling the
  transition to post-quantum cryptography. This proposal will focus on
  concrete cryptographic constructions and implementations of
  post-quantum cryptography (PQC) and secure distributed computation
  that can serve as inputs to standardisation processes or as
  contributions to widely deployed cryptographic libraries and
  protocol software stacks.
\end{abstract}

\section*{State of the art}

Formal methods progressed to an impressive level of maturity, and
several tools for systematically preventing entire classes of bugs in
cryptographic software now exist. Frameworks like F* [FSTR], EasyCrypt
[EC], and Coq [FIAT] are used to verify high-performance cryptographic
code written in C and assembly. Tools like CryptoVerif [CVRF] and
EasyCrypt are used to verify the correctness of crypto security
proofs. In practice, protocol stacks for TLS [TLS], Signal [SIGN], Key
Management [KMS] and crypto standards [SHA3,HACL] have been verified
with these tools.

This project will be focused on a tool-chain that includes EasyCrypt,
and the Jasmin language. EasyCrypt is an interactive proof assistant
for the verification of cryptographic security proofs. It adopts the
code-based approach, where primitives, security goals and hardness
assumptions are expressed as probabilistic programs. EasyCrypt offers
logics to reason about programs written in an probabilistic imperative
language, as well as establishing relations between two program
executions. The EasyCrypt logics have been successfully used to
machine-check a number of relevant cryptographic security proofs and
implementations [SHA3,EUC,MASK,DIFP]. Jasmin [JASM] is a programming
language designed to allow “assembly in the head” (a mixture of
high-level and low-level, platform-specific, programming); it is
supported by a formally verified (certified in Coq), predictable,
compiler which empowers programmers to write highly efficient
fine-tuned code. The generated verified assembly code is able to match
the performance of the best available implementations for the encoded
cryptographic primitive. Moreover, Jasmin code can be proved correct
and secure via an equivalence proof to a high-level specification in
EasyCrypt: specifically, the Jasmin compiler is able to create an
EasyCrypt translation of its source. End-to-end security and
correctness follow from the certification of the Jasmin compiler
(source-to-assembly) and from the EasyCrypt proof (source-to-spec).

Experience in using EasyCrypt and Jasmin [SoK] shows that both
foundational and applied research are needed to tackle two classes of
cryptographic protocols that have recently assumed a prominent
relevance due to advances of Quantum Computing: 1) post-quantum
cryptography (PQC) and 2) secure distributed (multiparty)
computation. PQC should not be confused with quantum computing or
quantum cryptography; its goal is to create cryptographic schemes that
can be used today and resist potential quantum attacks in the
future. PQC relies on different mathematical abstractions and
computational assumptions than classical crypto such as lattice-based
[CRYS] and isogeny-based assumptions [SIKE]. These imply reasoning
about distributions and complex mathematical objects for which little
has been done in the machine-checking setting; indeed, proof
techniques are still maturing in the cryptographic community to permit
fine-tuning parameter sizes and improve performance. The majority of
these proofs assume a quantum adversary interacting with a classical
system, which can be formalized as an extension of the current
EasyCrypt semantics, but this is currently lacking. Implementing these
primitives also raises new challenges for the Jasmin framework; e.g.,
it was not yet considered rejection sampling mechanisms, which are
crucial in PQC. Orthogonal challenges arise in the verification of
interactive protocols in Jasmin and EasyCrypt. Each cryptographic
primitive is proved secure in a security model that captures its use
in the real world. Primitives such as key exchange have been proved
secure in a variety of models [KECR], with surprising complexity for
two-party protocols: the goal is to capture concurrent executions and
deal with composition within the security proof itself. General
approaches to composability [UC] exist, but these exclude some of the
more efficient instantiations, or require ad-hoc adaptations to
capture the associated caveats.  Recent works [KMS,EUC,LMPC] have
highlighted a mismatch between the security semantics of multiparty
computation and the EasyCrypt framework. This is not surprising, as
main use cases for EasyCrypt are non-interactive primitives. What *is*
surprising is that it can used EasyCrypt to reason about restricted
classes of distributed protocols, such as those offering semi-honest
security, or constant-round two-party computation [LMPC, EUC]. It has
been explored more complex protocols [KMS], and identified the main
bottlenecks: when the scheduling of executions becomes even moderately
complex, the number of cases explodes and there is a large overhead in
specifying and verifying global invariants by hand. These results
point to new directions we aim to explore, while stressing the goal
being of obtaining verified proofs matching those written by
cryptographers, rather than using a symbolic model (e.g., as in
Tamarin or ProVerif), which are incomparable.

% HACSpec [HACS] is a common specification language for cryptography
% that can be used by technical standards, software developers, and feed
% formal verification tools. Syntactically, HACSpec is a subset of Rust,
% and hence is familiar to developers; most importantly for standards,
% specifications are executable, they can be tested for correctness and
% interoperability, and to generate test vectors. HACSpec comes with a
% translator tool that can feed various formal verification
% frameworks. By using a common specification language, formal
% verification can be done by larger teams working on their tools of
% choice. At the moment, there is only a very simple proof of concept
% for generating EasyCrypt code. PRESTO will contribute to the HACSpec
% development and ensure that the extensions we make to EasyCrypt and
% Jasmin can have a greater impact via integration into the HACSpec
% tool-chain.

%\bigskip

% Computer Aided Cryptography (CAC) aims to develop tools for the
% analysis and implementation of cryptographic protocols, including
% specification and implementation languages, compilers and formal
% verification tools. The formal verification dimension opens the way
% for High-Assurance Cryptographic Software, the overarching goal of
% this project.

% A recent SoK paper [SoK] published at IEEE S\&P 2021 summarises
% challenges in this area; two of these challenges serve as the main
% motivation for this project: 1) removing scalability bottlenecks in
% the analysis of distributed cryptographic protocols and 2) enabling
% the transition to post-quantum cryptography. These challenges shall be
% considered in the concrete setting of the tool-chain that combines
% HACSpec, Jasmin and EasyCrypt.

% \begin{itemize}
% \item HACSpec [HACS] is a new specification language for cryptography
%   aimed at non-experts in formal verification. HACSpec aspires to be a
%   common (executable) specification language that can be adopted in
%   cryptographic standards. Tool-specific specifications can be
%   automatically generated from a single HACSpec specification, so that
%   formal verification results obtained using different tools can be
%   compared and composed at the specification level.
% \item EasyCrypt [EC] is an interactive prover for cryptography. It
%   allows reasoning both about security proofs and the correctness of
%   cryptographic code using program logics, ranging from standard Hoare logic
%   to probabilistic relational Hoare logic over open programs (i.e.,
%   programs parametrized by adversarial algorithms). EasyCrypt has been
%   used to analyse a number of cryptographic constructions, including
%   high-profile industrial protocols and widely used cryptographic
%   standards.
% \item Jasmin [JASM] is an implementation language for high-speed
%   cryptographic code, which comes with a certified (verified in Coq)
%   compiler to assembly. Jasmin is formal verification friendly, as the
%   Jasmin compiler can also generate EasyCrypt descriptions of source
%   programs: Jasmin programs can be proved correct and secure by
%   relating them to specifications analysed in EasyCrypt and, in
%   particular, those generated from HACSpec tool-agnostic
%   specifications.
% \end{itemize}


\section*{Objectives}

The aim of this project is to develop extensions to the
Jasmin-EasyCrypt tool chain, so that it can be used to formally verify
the security and correctness of state of the art post-quantum
cryptography (PQC) implementations. For concreteness, most of the
effort will concentrate on the candidate submissions to the ongoing
NIST competition for PQC. Specifically, this project aims at
addressing the following set of challenges:
\begin{itemize}
\item To develop EasyCrypt libraries supporting for currently
  lacking data types and operators. To support the most efficient
  lattice-based schemes, there's need to cover polynomials over rings
  and finite fields, cyclotomic polynomials, matrices and vectors
  thereof. Important operations include the NTT transform, an analogue
  of the Fourier transform, norm computations, and sampling from
  non-uniform distributions, namely for dealing with low-norm
  noise. In parallel, Jasmin must be extended to deal with machine
  instructions that are sometimes used to optimize the implementations
  of these operations, namely (vectorized) floating point
  instructions.
\item To extend EasyCrypt with new program logics to deal with quantum
  attackers, i.e. an attacker in possession of a quantum computer (say
  a large organization or a country) is trying to break cryptography
  implemented in classical computers. This model is particularly
  important for long-term security of data protected today. There is a
  growing number of results [PQPR,PQRO] in cryptography that extend
  classical constructions and generic transformations (e.g., those
  based on random oracles) to this setting, and some seminal work in
  formalizing some of these results [PQEC]. Our goal will be to
  distill these developments into extended EasyCrypt and Why3 logics,
  to enable the machine-checking of formal security proofs of some of
  the NIST candidates.
\item To develop proof techniques and automation to deal with
  optimizations. PQC proofs are also challenging because they use
  aggressive parameter optimization in order to obtain a level of
  performance compatible with real-world use. This often implies
  adopting new proof techniques to improve the bounds, which deviate
  from the standard game-hopping approach. Two examples of this are
  the use of truncation and rounding to introduce noise and
  simultaneously compress public and secret keys. These techniques
  imply that schemes do not always work correctly, and it is necessary
  to perform intricate analysis of complex distributions to bound the
  probability of error. Similarly, improving bounds when adversaries
  interact with quantum random oracles and similar abstractions
  require reasoning about amortized bad event analysis, i.e., avoiding
  the use of coarse union bounds across potential occurrences when
  bounding the probability of a bad event.
\end{itemize}

\section*{Work description}

In what follows, we detail some of the activities planned in this
projects, and specifically focusing on the first year of the
doctoral programme.

The primary objective of the first year of the project is, of course,
formative. As such, it will be followed a comprehensive program of
supervised studies on foundational subjects in the area, such as
theoretical cryptography; required mathematical structures; and
deductive reasoning. Moreover, it is planned to attend some
international advanced schools in the area, such as the ``Summer
School on real-world crypto and privacy''
(\url{https://summerschool-croatia.cs.ru.nl/2022/}) and/or ``BIU
Winter School on Cryptography''
(\url{https://www.youtube.com/watch?v=fV3oD4sarNA}).

Having in mind that the main goal of the first year is formative, it
is nevertheless planned to start a \emph{seed project} to be developed
in parallel with the study mentioned above, and encompassing already some
of the ingredients that shall be addressed during the
research. This seed project will consist in implementing in Jasmin, and
experimenting with, different versions of the ZKBoo protocol [ZKBoo],
which is the basis of PQC Picnic signature scheme. The choice of the
aforementioned protocol is justified by the fact that it is based on
the so called MPC-in-the-Head transformation, which jointly require
reasoning about distributed protocols and post-quantum cryptography --
the main challenges addressed earlier. Moreover, it is used as a
sub-component for a signature scheme through the Fiat-Shamir
construction whose formalisation in EasyCrypt is, by itself, of
independent interest.  But even if this seed project already addresses
some of the central points of the research line planned for the PhD,
the main motivation for undertaking it is indeed still formative in
its nature. Specifically:
\begin{itemize}
\item the Picnic signature scheme was at the heart of the author's MSc
  thesis [MScThesis], and as such this implementation can be seen a
  natural follow-up of that work, benefiting from the author's
  familiarity with the subject;
\item it could, and should, be understood as an opportunity for an
  hands-on approach to the EasyCrypt/Jasmin development
  framework. Such a non-trivial development shall certainly demand a
  deep understanding of the underlying tools, allowing for a first
  assessment of what are their strengths and limitations in targeting
  the aimed schemes;
\item last but not least, it is also an opportunity to get knowledge
  and interact with the research community around the EasyCrypt/Jasmin
  framework, as well as the development teams of the tools.
\end{itemize}




\printbibliography{}

\end{document}
